\section{Evaluatie}
\noindent Het framework is ge\"installeerd in twee bestaande applicaties. Het eerste project is het Dodona project 
\footnote{\url{https://dodona.ugent.be/}} van de Universiteit Gent. Dit project is gebruikt om de performantie van de voorspellingsalgoritmen te evalueren. Aangezien de agent in deze masterproef enkel compatibel is met Java applicaties en Dodona gebouwd is in Ruby-on-Rails, is een tweede testapplicatie gebruikt om de installatie te verifi\"eren. Hiervoor werd het Stratego\footnote{Java Spring applicatie ontwikkeld voor het vak Software Engineering 2.} project gebruikt. Aansluitend beantwoordt deze masterproef drie onderzoeksvragen, met als doel waardevolle inzichten te vergaren over testpakketten.

\subsection{Performantie}
\noindent \Cref{tbl:performance-dodona} vergelijkt de vier besproken algoritmen op twee vlakken. Deze vlakken zijn respectievelijk het aantal uitgevoerde tests tot de eerste gefaalde test en de bijbehorende uitvoeringstijd van het parti\"ele testpakket. Deze resultaten tonen aan dat het Alfa algoritme bijna $\SI{30}{}$ keer minder tests uitvoert en dat de eerste falende test bijna onmiddellijk gedetecteerd wordt. Het Gretig en HGS-algoritme halveren het aantal uitgevoerde tests en zorgen tegelijk voor een sterke reductie van de uitvoeringstijd. De performantie van het ROCKET-algoritme is opmerkelijk. Dit algoritme voert veel meer tests uit dan de mediaan originele, niet-geprioriteerde volgorde, maar detecteert een falende test vier keer sneller.

\begin{table}[h]
	\centering
	\begin{tabularx}{\columnwidth}{|X||c|c|}
		\hline
		\textbf{Algoritme} & \textbf{Mediaan (tests)} & \textbf{Mediaan (tijd)}\\
		\hline
		\emph{Origineel} & $\SI{78}{}$ & $\SI{123}{\second}$\\
		
		\hline
		
		Alfa & $\SI{3}{}$ & $\SI{1}{\second}$\\
		
		Greedy & $\SI{33}{}$ & $\SI{12}{\second}$\\
		
		HGS & $\SI{10}{}$ & $\SI{6}{\second}$\\
		
		ROCKET & $\SI{170}{}$ & $\SI{32}{}$\\
		
		\hline
	\end{tabularx}
	\caption{Performantie op het Dodona project.}
	\label{tbl:performance-dodona}
\end{table}

\subsection{Onderzoeksvragen}
\noindent Deze masterproef beantwoordt drie onderzoeksvragen door gebruik te maken van gegevens van \travisci{}. Deze gegevens zijn verzameld en gepubliceerd door het TravisTorrent project \cite{msr17challenge} ($\SI{3702595}{}$ uitvoeringen) en door Durieux et al. \cite{travisanalysis} ($\SI{35793144}{}$ uitvoeringen).\\

\noindent \textbf{OV1: Kans op falen.}
De eerste onderzoeksvraag beschouwt de kans dat de uitvoering van een testpakket zal falen. Volgens beide bronnen is deze kans $\SI{15}{\percent}$.\\

\noindent \textbf{OV2: Gemiddelde uitvoeringstijd.}
Na het inspecteren van de TravisTorrent gegevens werd duidelijk dat de tijdsinformatie onvolledig en inaccuraat was, waardoor deze gegevens niet konden gebruikt worden om een betrouwbaar antwoord op deze vraag te bieden. De andere bron geeft aan dat \travisci{} hoofdzakelijk gebruikt wordt voor kleinere projecten, met een gemiddelde uitvoeringstijd van $\SI{385}{\second}$ per testpakket en een maximum van $\SI{26}{\hour}$.\\

\noindent \textbf{OV3: Opeenvolgend falen.}
De laatste onderzoeksvraag betreft de kans dat de tests van een project meer dan twee keer na elkaar falen. Hiervoor kan enkel de TravisTorrent bron worden gebruikt aangezien deze bron een koppeling bevat tussen alle uitvoeringen van hetzelfde project. Volgens deze bron is de kans op opeenvolgend falen $\SI{51.76}{\percent}$.