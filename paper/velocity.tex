% !TeX root = thesis.tex

\chapter{Proposed framework: \velocity{}}
\label{ch:velocity}
The implementation part of this thesis will provide a framework and a set of tools, tailored at optimising the test suite as well as providing accompanying metrics and insights. The framework was named \Gls{velocity} to reflect its purpose of enhancing the efficiency and speed of \CI{}. This paper will now proceed by first describing the design goals of the framework, after which a high-level schematic overview of the implemented architecture will be provided. The architecture consists of a seven-step pipeline, divided into three individual components. These steps will be elaborated in more detail in \cref{sec:velocity-pipeline}. Subsequently, the next section will present the \emph{Alpha} algorithm as as a novel prioritisation algorithm, and this chapter will be concluded with a presentation of the analysis tools.

% !TeX root = ../thesis.tex

\section{Design goals}
\velocity{} has been implemented with three design goals in mind:
\begin{enumerate}
	\item \textbf{Extensibility:} It should be possible and straightforward to support additional \CI{} systems, programming languages and test frameworks. Subsequently, a clear interface should be provided to integrate additional prioritisation algorithms.
	
	\item \textbf{Language agnosticism:} This design goal is related to the framework being extensible. The implemented tools should not need to be aware of the programming language of the source code, nor the used test framework.
	
	\item \textbf{Self-improvement:} The prioritisation framework supports all of the algorithms presented in \autoref{sec:relatedwork-algorithms}. It is possible that the performance of a given algorithm is strongly dependent on the nature of the project it is being applied to. In order to facilitate this behaviour, the framework should be able to measure the performance of every algorithm and ``learn'' which algorithm offers the best prediction, given a set of source code.
\end{enumerate}
\clearpage
% !TeX root = ../thesis.tex

\section{Architecture}

\begin{figure}[htbp!]
	\centering
	\includegraphics[height=\textheight]{assets/diagrams/sequence-diagram.pdf}
	\caption{Sequence diagram of \velocity{}}
	\label{fig:velocity-sequence-diagram}
\end{figure}

\subsection{Agent}
\label{ssec:velocity-frontend}
The first component that we will consider is the agent. The agent interacts directly with the source code and the test suite and is, therefore, the only component that is specific to the programming language and the test framework. Every programming language and test framework requires a different implementation of the agent, although these implementations are strongly related. This thesis provides a Java agent, which is available as a plugin for the Gradle and JUnit test framework, a combination which has previously been described in \cref{ssec:relatedwork-gradle-junit}. When the test suite is started, the plugin will contact the controller (\cref{ssec:velocity-controller}) to obtain the prioritised test case order and subsequently execute the test cases in that order. Afterwards, the plugin will send a feedback report to the controller, where it is analysed.

\subsection{Controller}\label{ssec:velocity-controller}
The second component is the core of the framework, acting as an intermediary between the agent on one side and the predictor (\cref{ssec:velocity-predictor}) on the other side. In order to satisfy the second design goal and as such allow language agnosticism, the controller exposes a \Gls{rest}-interface, to which the agent can communicate using the \texttt{HTTP} protocol. On the other side, the controller does not communicate directly with the predictor but stores prediction requests in a shared database instead. The predictor will periodically poll this database and update the request with the predicted order. Besides providing routing functionality between the agent and the predictor, the controller is additionally responsible for updating the meta predictor (\cref{ssec:pipeline-postanalysis}) by evaluating the accuracy of earlier predictions.

\subsection{Predictor and Metrics}\label{ssec:velocity-predictor}
The final component is the predictor. The predictor is responsible for applying the prioritisation algorithms to predict the optimal execution order of the test cases. This order is calculated by first executing ten prioritisation algorithms and subsequently consulting the meta predictor to determine the preferred sequence. The predictor has been implemented in Python, because of its accessibility and compatibility with various existing libraries such as NumPy\footnote{\url{https://numpy.org/}} and TensorFlow\footnote{\url{https://www.tensorflow.org/}}, to allow advanced prioritisation algorithms. 
\clearpage
% !TeX root = ../thesis.tex

\section{Pipeline}
 for interpreting the changed 



design goal 2 geimplementeerd via REST interface (leg rest uit)


This chapter will start by providing a high-level overview of the implemented architecture, followed by 

to optimise the test suite, as well as provide useful insights. The framework was named \emph{VeloCIty} to reflect its purpose of enhancing the speed of \CI{} systems. This chapter will start by 




(TODO)

- Implementatiedetails van algoritmes

- Uitwerking: nog onder voorbehoud (2e semester)

- Metapredictor: Voer alle algoritmes eens uit en rangschik ze volgens hoe goed ze het voorspeld hebben

- Scoringsmechanisme: Nog bepalen

- Junit: https://www.baeldung.com/junit-5-test-order

\subsection{Architectuur}
- Commit
- POST /commit
data: parent commit hash, huidige commit hash
result: volgorde van tests
- voer tests uit in gegeven volgorde
- POST /result
data: huidige commit hash, gefaalde tests

\subsection{Junit-reorder}
- Orde vastleggen in yaml files (plaats voorbeeld)
- Out of the box support om tests binnen eenzelfde klasse in volgorde uit te voeren, maar niet class-wide support
- Na elke uitvoering van een test wordt een nieuwe processor aangemaakt met een filter op de methodnaam, dit heeft als nadeel dat hergebruiken van initialisaties niet gaat. Misschien hier nog iets op vinden
- Parallelliseren momenteel niet gesupport, geen idee hoe dat zou werken met een volgorde (kan niet gewoon verdelen in volgorde want stel dat 1 test heel lang duurt dan is die volgorde niet meer juist)
- Shortcuts genomen wegens problemen met gradle (shaded jars) -> bepaalde JUnit functionaliteit weggegooid (vooral @Ignored annotatie)

Na elke uitvoering van test wordt coverage data bijgehouden om later te analyseren door test processor hack (maak hier zo'n mooi ding van).

Coverage data formaat optimaliseren zodat xmlfile niet gigantisch groot wordt -> done

Algoritmes aanpassen zodat branches ook rekening mee gehouden wordt (stel dat test A lijn 5 covert en test B ook, maar lijn 5 is een if-statement met 2 condities en test A triggert enkel de eerste conditie, dan zal test B met lagere prioriteit worden uitgevoerd of zelfs niet omdat de lijn al gecoverd is) -> eventueel via mutator die branches naar lijnen uitsplitst in zowel tests als source

Andere programmeertalen kunnen op zelfde manier door tests sequentieel uit te voeren

%<https://rubygems.org/gems/reverse_coverage/
\clearpage
% !TeX root = ../thesis.tex

\section{Alpha algorithm}
Besides the algorithms which have been presented in \autoref{sec:relatedwork-algorithms}, an additional algorithm has been implemented: the \emph{Alpha} algorithm. This was constructed by examining the philosophy behind every discussed algorithm and subsequently combining the best ideas into a novel prioritisation algorithm. The specification below will assume the same naming convention as described in \autoref{def:alg-naming}. The pseudocode is provided in Algorithm \ref{alg:alpha}\\

\noindent The algorithm takes the following inputs:
\begin{itemize}
	\item the set of all $n$ test cases: $TS = \{T_1, \dots, T_n\}$
	
	\item the set of $m$ \emph{affected} test cases: $AS = \{A_1, \dots, A_m\} \subseteq TS$. A test case $A$ is considered ``affected'' if any source code line which is covered by $A$ has been modified or removed in the commit that is being predicted.
	
	\item $C$: the set of all lines in the application source code, for which a test case $t \in TS$ exists that covers this line and that has not yet been prioritised. Initially, this set contains every covered source code line.
	
	\item the failure status of every test case, for every past execution out of $k$ executions of that test case: $F = \{F_1, \dots, F_n\}$, where $F_i = \{f_1, \dots, f_k\}$. $F_{tj} = 1$ implies that test case $t$ has failed in execution $j$.
	
	\item the execution time of test case $t \in TS$ for run $r \in [1 \dots k]$, in milliseconds: $D_{tr}$.
\end{itemize}

\noindent The first step of the algorithm is to determine the execution time $E_t$ of every test case $t$. This execution time is calculated as the average of the durations of every successful (i.e.) execution of $t$, since a test case will be prematurely aborted upon the first failed assertion, which introduces bias in the duration timings. In case $t$ has never been executed successfully, the average is computed over every execution of $t$.

\[
E_t = \left\{
\begin{array}{rl}
avg(\{D_{ti} \vert i \in [1 \dots k], F_{ti} = 0\}) & \exists j \in [1 \dots k] : F_{ti} = 0 \\
avg(\{D_{ti} \vert i \in [1 \dots k]\}) & \text{otherwise}
\end{array}
\right\}
\]

\noindent (TODO vermeld hier ergens bij dat eerste stap afgeleid is uit ROCKET)\\
\noindent (TODO vermeld ergens dat test groups beschouwd worden ipv individuele test cases zoals in HGS)\\

\noindent Next, the algorithm executes every affected test case that has also failed at least once in its three previous executions. This reflects the behaviour of a developer attempting to resolve the bug that caused the test case to fail. Specifically executing \emph{affected} failing test cases first is required in case multiple test cases are failing and the developer is solving resolving these one by one. In case there are multiple affected failing test cases, the test cases are prioritised according to ascending execution time. $C$ is updated after every considered test case.\\

\noindent Afterwards, the same step is repeated for every failed, but unaffected test case, again ordered by ascending execution time. Where the previous step helps developers to get fast feedback about whether or not the specific failing test case they were working on has been resolved, this step ensures that other failing test cases are not forgotten and are executed early in the run as well. After every prioritised test case, $C$ is updated by subtracting the code lines that are covered by at least one of these test cases.\\

(TODO)

\begin{itemize}
	\item Research (zie travistorrent in result sectie) heeft aangetoond dat maar bepaald percentage (reken uit; vrij laag) van test cases faalt $\Rightarrow$ In meeste situaties zal deze stap het meeste tijd in beslag nemen: voer alle affected test cases uit die niet gefaald zijn (want gefaalde zijn hierboven al uitgevoerd geweest), op volgorde van de resterende intersectie met C aangezien dat geupdate is geweest $\Rightarrow$ mogelijks schiet er niet eens meer een affected test case over ~ idee uit greedy algoritme
	
	\item Daarna, voer alle test cases uit die lijnen coveren die nog steeds niet gecoverd zijn op volgorde van grootste intersectie met C.
	
	\item Daarna, voer al de rest uit, als er nog iets is. $\Rightarrow$ Deze tests zijn eigenlijk redundant
\end{itemize}

\begin{algorithm}[h!]
	\caption{Alpha algorithm for \tcp{}}
	\label{alg:alpha}
	\begin{algorithmic}[1]
		\State TODO
	\end{algorithmic}
\end{algorithm}
% !TeX root = ../thesis.tex

\section{Analysis}
In this last section, we will take a look at the analytical features of the framework. Since the predictor already generates various statistics about the project which are required by the prioritisation algorithm, we can reuse these. The implementation of the analysis tool comprises a stand-alone version of the predictor daemon and supports the following six commands:

\paragraph*{\texttt{affected}:} The first command will determine which test cases have been affected by the changes in the given commit. This information is calculated based on the coverage information that the predictor has obtained from its last execution. \Cref{lst:analysis-affected} contains an example output of this command.

\begin{lstlisting}[language=bash, caption=Example output of the affected-command, label=lst:analysis-affected]
	(*\bfseries \$ predictor affected https://github.com/author/project f5a23e0*)
	FooTest.bar
	FooTest.foo
\end{lstlisting}

\paragraph*{\texttt{durations}:} The second command will compute the mean execution time of every test case in the repository and return the test cases from slowest to fastest.

\begin{lstlisting}[language=bash, caption=Example output of the durations-command, label=lst:analysis-durations]
	(*\bfseries \$ predictor durations https://github.com/author/project*)
	FooTest.foo: 200s
	OtherBarTest.bar: 100s
\end{lstlisting}

\paragraph*{\texttt{failures}:} Similar to the previous command, this command will determine the failure ratio of every test case in the repository. This ratio is equivalent to the number of failures, divided by the total amount of executions. Note that this denominator is not the same for every test case, since the test suite may be extended with new test cases. The output (\cref{lst:analysis-failures}) will list the test cases from the highest to the lowest failure rate.

\begin{lstlisting}[language=bash, caption=Example output of the failures-command, label=lst:analysis-failures]
	(*\bfseries \$ predictor failures https://github.com/author/project*)
	HelloWorldTest.hello: 25.00%
	FooBarTest.bar: 10.00%
\end{lstlisting}

\paragraph*{\texttt{predict}:} This command allows the user to invoke the predictor by hand for the given test run. We can use this to test new algorithms, as opposed to the usual predictor daemon which does not support repeated predictions of the same test run. The result will contain the prioritised order as predicted by every available algorithm. An example output of this command is listed in \cref{lst:analysis-predict}.

\begin{lstlisting}[language=bash, caption=Example output of the predict-command, label=lst:analysis-predict]
	(*\bfseries \$ predictor predict 1*)
	HGS: [FooTest.bar, OtherBarTest.bar, HelloWorldTest.hello]
	Alpha: [HelloWorldTest.hello, FooTest.bar, OtherBarTest.bar]
\end{lstlisting}

\paragraph*{\texttt{predictions}:} This command allows the user to retrieve historical prediction results. For deterministic algorithms,  this will result in the same output as the previous command (which will rerun the algorithms). However, since some algorithms contain a random factor, we do require a separate command that fetches the prediction of the given run from the database.

\begin{lstlisting}[language=bash, caption=Example output of the predictions-command, label=lst:analysis-predictions]
	(*\bfseries \$ predictor predictions 3*)
	HGS: [FooTest.bar, OtherBarTest.bar, HelloWorldTest.hello]
	AllRandom: [HelloWorldTest.hello, OtherBarTest.bar, FooTest.bar]
\end{lstlisting}
	
\paragraph*{\texttt{scores}:} The final command yields the current score of every prediction algorithm in the meta predictor table, for the given project. The predictor will always return the test sequence that has been predicted by the algorithm with the current highest score. In \cref{lst:analysis-scores} below, this would be the ROCKET algorithm.
	
\begin{lstlisting}[language=bash, caption=Example output of the scores-command, label=lst:analysis-scores]
	(*\bfseries \$ predictor scores https://github.com/author/project*)
	AllInOrder: -3
	ROCKET: 7
	GreedyCoverAffected: 4
\end{lstlisting}