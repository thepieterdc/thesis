\section{Techniques}
\noindent This thesis presents three techniques to solve this problem.

\begin{figure*}[t]
	\centering
	\subcaptionbox{Test Suite Minimisation}{\includegraphics[width=0.32\textwidth]{assets/tsm.tikz}}
	\hfill
	\subcaptionbox{Test Case Selection}{\includegraphics[width=0.32\textwidth]{assets/tcs.tikz}}
	\hfill
	\subcaptionbox{Test Case Prioritisation}{\includegraphics[width=0.32\textwidth]{assets/tcp.tikz}}
	\caption{Illustration of the techniques.}
\end{figure*}

\subsection{Test Suite Minimisation (TSM)}
\noindent The first technique is called Test Suite Minimisation (TSM) \cite{10.1002/stv.430}. TSM will attempt to reduce the size of the test suite by permanently removing redundant test cases according to the following definition:\\

\noindent\fbox{\begin{minipage}{\columnwidth}
\textbf{Given:}
\begin{itemize}
	\item $T = \{t_1, \dots, t_n\}$ a test suite consisting of test cases $t_j$.
	\item $R = \{r_1, \dots, r_m\}$ a set of requirements that must be satisfied in order to provide the desired ``adequate'' testing of the program.
	\item $\{T_1, \dots, T_m\}$ subsets of test cases in $T$, one associated with each of the requirements $r_i$, such that any one of the test cases $t_j \in T_i$ can be used to satisfy requirement $r_i$.
\end{itemize}
\mbox{}\\
TSM is then defined as the task of finding a subset $T'$ of test cases $t_j \in T$ that satisfies every requirement $r_i$.
\end{minipage}}

\subsection{Test Case Selection (TCS)}
\noindent Instead of permanently removing redundant test cases, it is also possible to analyse the code changes and deduce which test cases should be executed and which might ones be omitted altogether. This technique is referred to as Test Case Selection (TCS) \cite{10.1002/stv.430}.\\

\noindent\fbox{\begin{minipage}{\columnwidth}
\textbf{Given:}
\begin{itemize}
	\item $T$ the test suite.
	\item $P$ the previous version of the codebase.
	\item $P'$ the current (modified) version of the codebase.
\end{itemize}
\mbox{}\\
TCS aims to find a subset $T' \subseteq T$ that is used to test $P'$. 
\end{minipage}}

\subsection{Test Case Prioritisation (TCP)}
\noindent While TSM and TCS execute only the relevant test cases, sometimes it might be required that every test case is successfully executed. Consider, for example, critical software for medical purposes. In this case, it is still possible to optimise the test suite by executing all test cases in a specific order. Test Case Prioritisation (TCP) \cite{10.1002/stv.430} constructs an ordered sequence that aims to fulfil a predetermined objective as fast as possible. This thesis primarily uses the detection of the first failed test case as the objective.\\

\noindent\fbox{\begin{minipage}{\columnwidth}
\textbf{Given:}
\begin{itemize}
	\item $T$ the test suite.
	\item $PT$ the set of permutations of $T$.
	\item $f: PT \mapsto \mathbb{R}$ a function from a subset to a real number, this function is used to compare sequences of test cases to find the optimal permutation.
\end{itemize}
\mbox{}\\
TCP finds a permutation $T' \in PT$ such that $\forall T'' \in PT : f(T') \ge f(T'') \Rightarrow (T'' \ne T')$. 
\end{minipage}}