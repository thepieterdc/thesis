\section{Framework}
% Alpha naam hierbijvermelden.
\noindent This thesis proposes \velocity{} as a language-agnostic framework that enables Test Case Prioritisation on existing software projects. The architecture consists of three main components, followed by a meta predictor and a novel prioritisation algorithm.

\subsection{Agent}
\noindent The first component is the agent. This component hooks into the testing framework of the application to execute the test cases in the required order. The agent obtains the optimal execution sequence by communicating with the next component, which is the controller.

\subsection{Controller}
\noindent The controller is a daemon which performs two tasks. First, the controller listens for requests from agents and acts as a relay to the predictor daemon. Additionally, the controller receives feedback from the agent after every executed test run. This information is used to update the meta predictor, which will be described later.

\subsection{Predictor}
\noindent The final main component of the architecture is the predictor daemon. This component is responsible for interpreting the code changes and determining the optimal order of execution using ten built-in prediction algorithms. These algorithms are variations of the three discussed algorithms, as well as the Alpha algorithm.

\subsection{Meta predictor}
\noindent Since the predictor daemon contains multiple prediction algorithms, a small extra component is required to decide which sequence should be preferred as the final execution order. The meta predictor is a table which assigns a score to every prediction algorithm. The final execution order is the one that has been predicted by the prediction algorithm with the highest score. The controller evaluates the performance of every algorithm in the feedback phase, as mentioned before, and updates the scores accordingly.

\subsection{Alpha algorithm}
\noindent In addition to the Greedy, HGS and ROCKET algorithms, the framework features a custom prioritisation algorithm as well. This algorithm starts by inspecting the changed code lines to obtain the affected test cases $ATS$. Among $ATS$, the algorithm selects every test case that has failed at least once in its previous three executions and sorts those by increasing execution time. Next, the algorithm selects the remaining test cases from $ATS$ and sorts those equivalently. After these two steps, the algorithm proceeds like the greedy algorithm until it has processed every test case.