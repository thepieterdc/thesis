% !TeX root = extended-abstract.tex

\section{Introduction}
\noindent The most characteristic trait of modern times is the astonishing speed at which everything in this world is evolving. This statement holds in particular for the field of computer science, where new technologies emerge almost every single day. While these inventions often seem primarily hardware-related, such as smartwatches, self-driving cars, or even biological technology, they cannot function without an even more sophisticated software component that controls them. As such, both the complexity and size of software have grown exponentially over the last decades.\\

\noindent Software engineers have experienced that the traditional software methodologies are no longer sufficient to handle this fast-paced development and, as a result, have shifted towards other strategies. Instead of implementing the entire application at once, developers now prefer the Agile approaches \cite{beck2001agile}. The Agile methodologies depict that a minimal version of the application should be released as soon as possible, to reduce the financial risks taken. Afterwards, the functionality is extended incrementally. A report of The Standish Group confirms that the adoption of these new approaches has led to an increased success rate for new projects \cite{standish2015chaos}.\\

\noindent However, this evolution is not necessarily a good unfolding. Managing to complete a project is one thing, guaranteeing it is built reliably is a different matter. An increase in the complexity of software inevitably makes the software more error-prone. As an attempted solution, the Agile approaches propose Continuous Integration (CI) \cite{SmartJenkinsDefinitive}. This practice requires that the test suite of the application is executed after every code change (\Cref{fig:ext-en-ci}). Multiple CI-services have been created to assist in this process by means of automation. Optionally, every passed version can automatically be released to the end-users (Continuous Deployment).

\begin{figure}[h!]
	\centering
	\includegraphics[width=0.96\columnwidth]{assets/ci.tikz}
	\caption{Continuous Integration.}
	\label{fig:ext-en-ci}
\end{figure}

\noindent Yet, in the long run, a new problem will emerge. As mentioned before, the size of applications increases exponentially. Every change in the code must be followed by the introduction of one or more test cases to guarantee the correctness. Therefore, the size of the test suite will increase even faster than the size of the project itself, which leads to severe scalability issues.