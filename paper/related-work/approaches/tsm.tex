% !TeX root = ../../thesis.tex

\subsection{\tsm{}}
\label{ssec:tsm}
\tsm{}, also referred to as \emph{Test Suite Reduction}, aims to reduce the size of the test suite by permanently removing redundant tests. This problem is formally defined by Rothermel in \autoref{def:tsm} \cite{10.1002/stv.430}. 

\begin{definition}[\tsm{}]
\label{def:tsm}
\mbox{}\\Given:
\begin{itemize}
	\item $T = \{t_1, \dots, t_n\}$ a test suite consisting of tests $t_j$.
	\item $R = \{r_1, \dots, r_n\}$ a set of requirements that must be satisfied in order to provide the desired ``adequate'' testing of the program.
	\item $\{T_1, \dots, T_n\} \subseteq T$ subsets of test cases, one associated with each of the requirements $r_i$, such that any one of the test cases $t_j \in T_i$ can be used to satisfy requirement $r_i$.
\end{itemize}

\noindent \tsm{} is then defined as the task of finding a set $T'$ of test cases $t_j \in T$ that satisfies all requirements $r_i$.
\end{definition}

\noindent If we apply this definition to the concepts introduced in \autoref{chap:software-engineering}, the requirements $R$ can be interpreted as lines in the codebase that must be covered. With respect to the definition, a requirement can be satisfied by any test $t_j$ that belongs to subset $T_i$ of $T$. Observe that the problem of finding $T'$ is closely related to the \emph{hitting set problem} (\autoref{def:hitting-set}) \cite{10.1002/stv.430}.

\begin{definition}[Hitting Set Problem]
\label{def:hitting-set}
\mbox{}\\Given:
\begin{itemize}
	\item $S = \{s_1, \dots, s_n\}$ a finite set of elements.
	\item $C = \{c_1, \dots, c_n\}$ a collection of sets, with $\forall c_i \in C : c_i \subseteq S$.
	\item $K$ a positive integer, $K \le |S|$.
\end{itemize}

\noindent The hitting set is a subset $S' \subseteq S$ such that $S'$ contains at least one element from each subset in $C$.
\end{definition}

\noindent In the context of \tsm{}, $T'$ is precisely the hitting set of $T_i$s. In order to effectively minimise the amount of tests in the test suite, $T'$ should be the minimal hitting set \cite{10.1002/stv.430}, which is an NP-complete problem as it can be reduced to the \emph{Vertex Cover}-problem \cite{10.5555/574848}.