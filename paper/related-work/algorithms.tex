% !TeX root = ../thesis.tex

\section{Algorithms}
\label{sec:relatedwork-algorithms}
\acrshort{tcp} is essentially an extended version of \acrshort{tsm} since we can first execute the minimised test suite and afterwards the remaining test cases. Additionally, \cref{ssec:tsm} has explained that \acrshort{tsm} is an instance of the minimal hitting set problem, which is an NP-complete problem. Consequently, we know that both \acrshort{tsm} and \acrshort{tcp} are NP-complete problems as well and therefore, we require the use of \emph{heuristics}. A heuristic is an experience-based method that can be applied to solve a hard to compute problem by finding a fast approximation \cite{6588537}. However, the found solution will mostly be suboptimal, or sometimes the algorithm might even fail to find any solution at all. Given the relation between \acrshort{tsm} and the minimal hitting set problem, we can implement an optimisation algorithm by modifying any known heuristic that finds the minimal hitting set. This paper will now proceed by discussing a selection of these heuristics. The used terminology and the names of the variables have been changed to ensure mutual consistency between the algorithms. Every algorithm has been adapted to adhere to the conventions provided in \cref{def:alg-naming,def:cardinality}.

\begin{definition}[Naming convention]
\label{def:alg-naming}
\mbox{}
\begin{itemize}
	\item $TS = \{T_1, \dots, T_n\}$: the set of all test cases $t$ in the test suite.
	\item $RS = \{T_1, \dots, T_n\} \subseteq TS$: the representative set of test cases $t$ that have been selected by the algorithm.
	\item $C = \{c_1, \dots, c_m\}$: the set of all source code lines in the application, that are covered by at least one test case $T \in TS$.
	\item $CT = \begin{bmatrix}
		CT_1 & \dots & CT_m
	\end{bmatrix}$: the list of test groups.
		\begin{itemize}
			\item $CT_c = \{T_1, \dots, T_n\} \subseteq TS$: the test group $c$, which corresponds to the set of all test cases $T \in TS$ that cover the source code line $c \in C$.
		\end{itemize}
	\item $TL = \begin{bmatrix}
		TL_1 & \dots & TL_n
	\end{bmatrix}$: the list of coverage groups.
		\begin{itemize}
			\item $TL_t = \{c_1, \dots, c_m\} \subseteq C$: the set of all source code lines $c \in C$ that are covered by test case $t \in TS$.
		\end{itemize}
\end{itemize}
\end{definition}

\begin{definition}[Cardinality]
\label{def:cardinality}
For a finite set $S$, the cardinality $|S|$ is defined as the number of elements in $S$. In case of potential confusion, we can use $Card(S)$ to denote the cardinality of $S$.
\end{definition}

% !TeX root = ../../thesis.tex

\subsection{Greedy algorithm}
\label{ssec:alg-greedy}
The first algorithm is a \emph{greedy} heuristic, which was originally designed by Chvatal to find an approximation for the set-covering problem \cite{evaluationoftestsuiteminimization}. A greedy algorithm always makes a locally optimal choice, assuming that this will eventually lead to a globally optimal solution \cite{10.5555/1614191}. Algorithm \ref{alg:tsm-greedy} presents the Greedy algorithm for \tsm{}. The goal of the algorithm is to construct a set of test cases that cover every line in the code, by requiring as few tests as possible.\\

\noindent Initially, the algorithm starts with an empty result set $RS$, the set $TS$ of all test cases and the set $C$ of all coverable source code lines. Furthermore, $TL_t$ denotes the set of source code lines in $C$ that are covered by test case $t \in TS$. Subsequently, the algorithm iteratively selects test cases from $TS$ and adds them to $RS$. The locally optimal choice is to always select the test case that will contribute the most still uncovered lines, ergo the test $t$ for which the cardinality of the intersection between $C$ and $TL_t$ is maximal. After every iteration, the now covered lines $TL_t$ are removed from $C$ and the selection process is repeated until $C$ is empty. Upon running the tests, only the tests in $RS$ must be executed. This algorithm can be converted to make it applicable to \tcp{} by converting the set $RS$ to a list to maintain the order in which the test cases were selected, which is equivalent to the prioritised order of execution.

\begin{algorithm}[h!]
\caption{Greedy algorithm for \tsm{}}
\label{alg:tsm-greedy}
\begin{algorithmic}[1]
	\State {\bfseries Input:} Set $TS$ of all test cases, set $C$ of all source code lines that are covered by any $t \in TS$ and $TL_t$ the set of all lines are covered by test case $t \in TS$.
	\State {\bfseries Output:} Subset $RS \subseteq TS$ of tests to execute.
	\State $RS \gets \emptyset$
	\While{$C \neq \emptyset$}
		\State $t\_max \gets 0$
		\State $tl\_max \gets \emptyset$
		
		\ForAll{$t \in TS$}
			\State $tl\_current \gets C \cap TL_{t}$
			\If{$|tl\_current| > |tl\_max|$}
				\State $t\_max \gets t$
				\State $tl\_max \gets tl\_current$
			\EndIf
		\EndFor
		
		\State $RS \gets RS \cup \{t\_max\}$
		\State $C \gets C \setminus tl\_max$
	\EndWhile
\end{algorithmic}
\end{algorithm}

\clearpage
% !TeX root = ../../thesis.tex

\subsection{HGS}\label{ssec:alg-hgs}
The second algorithm was created by Harrold, Gupta and Soffa \cite{hgs}. This algorithm constructs the minimal hitting set of the test suite in an iterative fashion. As opposed to the greedy algorithm (\autoref{ssec:alg-greedy}), the HGS algorithm considers the test groups $CT$ instead of the set $TLt$ to obtain a list of test cases that cover all source code lines. More specifically, this algorithm considers the distinct test groups, denoted as $CTD$. Two test groups are considered indistinct if they differ in at least one test case. The pseudocode for this algorithm is provided in Algorithm \autoref{alg:hgs}.\\

\noindent Similar to the previous algorithm, an empty representative set $RS$ is constructed in which the selected test cases will be stored. The algorithm begins by iterating over every source code line $l \in C$ and constructing the corresponding set of test groups $CT_l$. As mentioned before, for performance reasons this set is reduced to $CTD$, only retaining distinct test groups. Next, the algorithm selects every test group of which the cardinality is equal to 1 and adds these to $RS$. This corresponds to every test case that covers a line of code, which is exclusively covered by that single test case. Subsequently, the lines that are covered by any of the selected test cases are removed from $C$. This process is repeated for an incremented cardinality, until every line in $C$ is covered. Since the remaining test groups will now contain more than one test case, the algorithm needs to make a choice on which test case to select. The authors have chosen that the test case that occurs in the most test groups is preferred. In the event of a tie, this choice is deferred until the next iteration.\\

\noindent The authors have provided an accompanying calculation of the computational time complexity of this algorithm \cite{hgs}. With respect to the naming convention introduced in \autoref{def:alg-naming}, additionally let $n$ denote the number of distinct test groups $CTD$, $nt$ the number of test cases $t \in TS$ and $MAX\_CARD$ the cardinality of the largest test group. The HGS algorithm consists of two steps which are performed repeatedly. The first step involves computing the number of occurrences of every test case $t$ in each test group. Given that there are $n$ distinct test groups and, in the worst case scenario, each test group can contain $MAX\_CARD$ test cases which all need to be examined once, the computational cost of this step is equal to $O(n * MAX\_CARD)$. In order to determine which test case should be included in the representative set $RS$, the algorithm needs to find all test cases for which the number of occurrences in all test groups is maximal, which requires at most $O(nt * MAX\_CARD)$. Since every repetition of these two steps adds a test case that belongs to at least one out of $n$ test groups to the representative set, the overall runtime of the algorithm is $O(n * (n + nt) * MAX\_CARD)$.
\begin{algorithm}[h!]
\caption{HGS algorithm (\cite{hgs})}
\label{alg:hgs}
\begin{algorithmic}[1]
	\State {\bfseries Input:} Distinct test groups $T_1, \dots T_n \in CDT$, containing test cases from $TS$.
	\State {\bfseries Output:} Subset $RS \subseteq TS$ of tests to execute.
	\State $marked \gets array[1 \dots n]$ \Comment{initially $false$}
	\State $MAX\_CARD \gets max \{Card(T_i) \vert T_i \in CDT\}$
	\State $RS \gets \bigcup \{ T_i \vert Card(T_i) = 1 \}$
	\ForAll{$T_i \in CDT$}
		\If{$T_i \cap RS \neq \emptyset$} $marked[i] \gets true$ \EndIf
	\EndFor
	\State $current \gets 1$
	\While{$current < MAX\_CARD$}
		\State $current \gets current + 1$
		\While{$\exists T_i : Card(T_i) = current, marked[i] = false$}
			\State $list \gets \{t \vert t \in T_i : Card(T_i) = current, marked[i] = false\}$
			\State $next \gets SelectTest(current, list)$
			\State $reduce \gets false$
			\ForAll{$T_i \in CDT$}
				\If{$next \in T_i$}
					\State $marked[i] = true$
					\If{$Card(T_1) = MAX\_CARD$} $reduce \gets true$ \EndIf
				\EndIf
			\EndFor
			\If{$reduce$}
				\State $MAX\_CARD \gets max \{Card(T_i) \vert marked[i] = false\}$
			\EndIf
			\State $RS \gets RS \cup \{next\}$
		\EndWhile
	\EndWhile
	
	\Function{SelectTest}{$size$, $list$}
		\State $count\gets array[1 \dots nt]$
		
		\ForAll{$t \in list$}
			\State $count[t] \gets |\{T_j \vert t \in T_j, marked[T_j] = false, Card(T_j) = size\}|$
		\EndFor
		
		\State $tests \gets \{t \vert t \in list, count[t] = max(count) \}$
		
		\If{$|tests| = 1$} \Return $tests[0]$
		\ElsIf{$|tests| = MAX\_CARD$} \Return $tests[0]$
		\Else{} \Return $SelectTest(size+1, tests)$
		\EndIf
	\EndFunction
\end{algorithmic}
\end{algorithm}
\clearpage
% !TeX root = ../../thesis.tex

\subsection{ROCKET algorithm}
\label{ssec:alg-rocket}

The third and final algorithm is the ROCKET algorithm. This algorithm has been presented by Marijan, Gotlieb and Sen \cite{6676952} as part of a case study to improve the testing efficiency of industrial video conferencing software. Contrarily to the previous algorithms, which attempted to execute as few test cases as possible, this algorithm does execute the entire test suite. Unlike the previous algorithms that only take code coverage into account, this algorithm also considers historical failure data and test execution time. The objective of this algorithm is twofold: select the test cases with the highest successive failure rate, while also maximising the number of executed test cases in a limited time frame. In the implementation below, we will consider an infinite time frame as this is a domain-specific constraint and irrelevant for this thesis. This algorithm will yield a total ordering of all the test cases in the test suite, ordered using a weighted function.\\

\noindent The modified version of the algorithm (of which the pseudocode is provided in \Cref{alg:rocket}) takes three inputs:
\begin{itemize}
	\item $TS = \{T_1, \dots, T_n\}$: the set of test cases to prioritise.
	\item $E = \begin{bmatrix}
		E_1 & \dots & E_n
	\end{bmatrix}$: the execution time of each test case.
	\item $F = \begin{bmatrix}
		F_1 & \dots & F_n
	\end{bmatrix}$: the failure statuses of each test case.
		\begin{itemize}
			\item $F_t = \begin{bmatrix}
				f_1 & \dots & f_m
			\end{bmatrix}$: the failure status of test case $t$ over the previous $m$ successive executions. $F_{ij} = 1$ if test case $i$ has failed in execution $(current - j)$, $0$ if it has passed.
		\end{itemize}
\end{itemize}

\noindent The algorithm starts by creating an array $P$ of length $n$, which contains the priority of each test case. The priority of each test case is initialised at zero. Next, we construct an $m \times n$ failure matrix $MF$ and fill it using the following formula.
\[
	MF[i, j] = \left\{
	\begin{array}{rl}
		1 & \text{if } F_{ji} = 1 \\
		-1 & \text{otherwise} \\
		\end{array}
	\right.
\]

\noindent \Cref{tbl:rocket-failurematrix} contains an example of this matrix $MF$. In this table, we consider the hypothetical failure rates of the last two executions of six test cases.

\begin{table}[h]
\centering
\begin{tabular}{| l || c | c | c | c | c | c |}
	\hline
	\textbf{run} & \textbf{$T_1$} & \textbf{$T_2$} & \textbf{$T_3$} & \textbf{$T_4$} & \textbf{$T_5$} & \textbf{$T_6$}\\\hline
	$current - 1$ & $1$ & $1$ & $1$ & $1$ & $-1$ & $-1$\\
	$current - 2$ & $-1$ & $1$ & $-1$ & $-1$ & $1$ & $-1$\\
	\hline
\end{tabular}
\caption{Example of the failure matrix $MF$.}
\label{tbl:rocket-failurematrix}
\end{table}

\noindent Afterwards, we fill $P$ with the cumulative priority of each test case. We can calculate the priority of a test case by multiplying its failure rate with a domain-specific weight heuristic $\omega$. This heuristic reflects the probability of repeated failures of a test case, given earlier failures. In their paper \cite{6676952}, the authors apply the following weights:

\[
	\omega_i = \left.
	\begin{cases}
		0.7 & \text{if } i = 1 \\
		0.2 & \text{if } i = 2 \\
		0.1 & \text{if } i >= 3 \\
	\end{cases}
	\right.
\]
$$P_j = \sum_{i = 1 \dots m} MF[i, j] * \omega_i$$

\noindent Finally, the algorithm groups test cases based on their calculated priority in $P$. Every test case that belongs to the same group is equally relevant for execution in the current test run. However, within every test group, the test cases will differ in execution time $E$. The final step is to reorder test cases that belong to the same group in such a way that test cases with a shorter duration are executed earlier in the group.

\begin{algorithm}[h!]
\caption{ROCKET algorithm}
\label{alg:rocket}
\begin{algorithmic}[1]
	\State {\bfseries Input:} Set $TS = \{T_1, \dots, T_n\}$ of all test cases,

	Execution time $E_t$ of every test case,
	
	Failure status $F$ for each test case over the previous $m$ successive iterations.
	\State {\bfseries Output:} Priority of test cases $P$.
	\State $P \gets array[1 \dots n]$ \Comment{initially $0$}
	\State $MF \gets array[1 \dots m]$
	\ForAll{$i \in 1 \dots m$}
		\State $MF[i] \gets array[1 \dots n]$
		\ForAll{$j \in 1 \dots n$}
			\If{$F[j][i] = 1$}
				$MF[i][j] \gets -1$
			\Else{}
				$MF[i][j] \gets 1$
			\EndIf
		\EndFor
	\EndFor
	\ForAll{$j \in 1 \dots n$}
		\ForAll{$i \in 1 \dots m$}
			\If{$i = 1$}
				$P[j] \gets P[j] + (MF[i][j] * 0.7)$
			\ElsIf{$i = 2$}
				$P[j] \gets P[j] + (MF[i][j] * 0.2)$
			\Else{}
				$P[j] + (MF[i][j] * 0.1)$
			\EndIf
		\EndFor
	\EndFor
	\State $Q \gets \{P[j] \vert j \in 1 \dots n\}$ \Comment{distinct priorities}
	\State $G \gets array[1 \dots Card(Q)]$ \Comment{initially empty sets}
	\ForAll{$j \in 1 \dots n$}
		\State $p \gets P[j]$
		\State $G[p] \gets G[p] \cup \{j\}$
	\EndFor
	\State Sort every group in $G$ based on ascending execution time in $E$.
	\State Sort $P$ according to which group it belongs and its position within that group.
\end{algorithmic}
\end{algorithm}
