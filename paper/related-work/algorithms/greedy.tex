% !TeX root = ../../thesis.tex

\subsection{Greedy algorithm}\label{ssec:alg-greedy}
The first algorithm is a \emph{greedy} heuristic, which was initially designed by Chvatal to find an approximation for the set-covering problem \cite{evaluationoftestsuiteminimization}. A greedy algorithm always makes a locally optimal choice, assuming that this will eventually lead to a globally optimal solution \cite{10.5555/1614191}. \Cref{alg:tsm-greedy} presents the Greedy algorithm for \tsm{}. The objective of the algorithm is to construct a set of test cases that cover every line in the code, by requiring as few test cases as possible.\\

\noindent Initially, the algorithm starts with an empty representative set $RS$, the set $TS$ of all test cases and the set $C$ of all coverable source code lines. Furthermore, $TL$ denotes the set of coverage groups as specified in the definition. In essence, the algorithm will iteratively select test cases from $TS$ and add them to $RS$. The locally optimal choice is always to select the test case that will contribute the most still uncovered lines, ergo the test case $t$ for which the cardinality of the intersection between $C$ and $TL_t$ is maximal. After every iteration, we remove the code lines $TL_t$ from $C$, since these are now covered. We repeat this selection process until $C$ is empty, which indicates that we have covered every source code line. Afterwards, when we execute the test suite, we only need to execute test cases in $RS$. We can apply this algorithm to \tcp{} as well, by changing the type of $RS$ to a list instead. We require a list to maintain the insertion order since this is equivalent to the ideal order of execution.

\begin{algorithm}[h!]
\caption{Greedy algorithm for \tsm{}}
\label{alg:tsm-greedy}
\begin{algorithmic}[1]
	\Require{the test suite $TS$, all coverable lines $C$, the list of coverage groups $TL$}
	\Ensure{representative set $RS \subseteq TS$ of test cases to execute}
	\Procedure{GreedyTSM}{$TS, C, TL$}
		\State $RS \gets \emptyset$
		\While{$C \neq \emptyset$}
			\State $t\_max \gets 0$
			\State $tl\_max \gets \emptyset$
			
			\ForAll{$t \in TS$}
				\State $tl\_current \gets C \cap TL[t]$
				\If{$|tl\_current| > |tl\_max|$}
					\State $t\_max \gets t$
					\State $tl\_max \gets tl\_current$
				\EndIf
			\EndFor
			
			\State $RS \gets RS \cup \{t\_max\}$
			\State $C \gets C \setminus tl\_max$
		\EndWhile
		\State \Return $RS$
	\EndProcedure
\end{algorithmic}
\end{algorithm}
