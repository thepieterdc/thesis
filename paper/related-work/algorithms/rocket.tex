% !TeX root = ../../thesis.tex

\subsection{ROCKET algorithm}
\label{ssec:alg-rocket}

The third and final algorithm is the ROCKET algorithm. This algorithm has been presented by Marijan, Gotlieb and Sen \cite{6676952} as part of a case study to improve the testing efficiency of industrial video conferencing software. Contrarily to the previous algorithms, which attempted to execute as few test cases as possible, this algorithm does execute the entire test suite. Unlike the previous algorithms that only take code coverage into account, this algorithm also considers historical failure data and test execution time. The objective of this algorithm is twofold: select the test cases with the highest successive failure rate, while also maximising the number of executed test cases in a limited time frame. In the implementation below, we will consider an infinite time frame as this is a domain-specific constraint and irrelevant for this thesis. This algorithm will yield a total ordering of all the test cases in the test suite, ordered using a weighted function.\\

\noindent The modified version of the algorithm (of which the pseudocode is provided in \Cref{alg:rocket}) takes three inputs:
\begin{itemize}
	\item $TS = \{T_1, \dots, T_n\}$: the set of test cases to prioritise.
	\item $E = \begin{bmatrix}
		E_1 & \dots & E_n
	\end{bmatrix}$: the execution time of each test case.
	\item $F = \begin{bmatrix}
		F_1 & \dots & F_n
	\end{bmatrix}$: the failure statuses of each test case.
		\begin{itemize}
			\item $F_t = \begin{bmatrix}
				f_1 & \dots & f_m
			\end{bmatrix}$: the failure status of test case $t$ over the previous $m$ successive executions. $F_{ij} = 1$ if test case $i$ has failed in execution $(current - j)$, $0$ if it has passed.
		\end{itemize}
\end{itemize}

\noindent The algorithm starts by creating an array $P$ of length $n$, which contains the priority of each test case. The priority of each test case is initialised at zero. Next, we construct an $m \times n$ failure matrix $MF$ and fill it using the following formula.
\[
	MF[i, j] = \left\{
	\begin{array}{rl}
		1 & \text{if } F_{ji} = 1 \\
		-1 & \text{otherwise} \\
		\end{array}
	\right.
\]

\noindent \Cref{tbl:rocket-failurematrix} contains an example of this matrix $MF$. In this table, we consider the hypothetical failure rates of the last three executions of six test cases.\\

\begin{table}[h]
\centering
\begin{tabular}{| l || c | c | c | c | c | c |}
	\hline
	\textbf{run} & \textbf{$T_1$} & \textbf{$T_2$} & \textbf{$T_3$} & \textbf{$T_4$} & \textbf{$T_5$} & \textbf{$T_6$}\\\hline
	$current - 1$ & $1$ & $1$ & $1$ & $1$ & $-1$ & $-1$\\
	$current - 2$ & $-1$ & $1$ & $-1$ & $-1$ & $1$ & $-1$\\
	$current - 3$ & $1$ & $1$ & $-1$ & $1$ & $1$ & $-1$\\
	\hline
\end{tabular}
\caption{Example of the failure matrix $MF$.}
\label{tbl:rocket-failurematrix}
\end{table}

\noindent Afterwards, we fill $P$ with the cumulative priority of each test case. We can calculate the priority of a test case by multiplying its failure rate with a domain-specific weight heuristic $\omega$. This heuristic reflects the probability of repeated failures of a test case, given earlier failures. In their paper \cite{6676952}, the authors apply the following weights:

\[
	\omega_i = \left.
	\begin{cases}
		0.7 & \text{if } i = 1 \\
		0.2 & \text{if } i = 2 \\
		0.1 & \text{if } i >= 3 \\
	\end{cases}
	\right.
\]
$$P_j = \sum_{i = 1 \dots m} MF[i, j] * \omega_i$$

\noindent Finally, the algorithm groups test cases based on their calculated priority in $P$. Every test case that belongs to the same group is equally relevant for execution in the current test run. However, within every test group, the test cases will differ in execution time $E$. The final step is to reorder test cases that belong to the same group in such a way that test cases with a shorter duration are executed earlier in the group.

\begin{algorithm}[h!]
\caption{ROCKET algorithm}
\label{alg:rocket}
\begin{algorithmic}[1]
	\State {\bfseries Input:} Set $TS = \{T_1, \dots, T_n\}$ of all test cases,

	Execution time $E_t$ of every test case,
	
	Failure status $F$ for each test case over the previous $m$ successive iterations.
	\State {\bfseries Output:} Priority of test cases $P$.
	\State $P \gets array[1 \dots n]$ \Comment{initially $0$}
	\State $MF \gets array[1 \dots m]$
	\ForAll{$i \in 1 \dots m$}
		\State $MF[i] \gets array[1 \dots n]$
		\ForAll{$j \in 1 \dots n$}
			\If{$F[j][i] = 1$} $MF[i][j] \gets -1$
			\Else{} $MF[i][j] \gets 1$
			\EndIf
		\EndFor
	\EndFor
	\ForAll{$j \in 1 \dots n$}
		\ForAll{$i \in 1 \dots m$}
			\If{$i = 1$} $P[j] \gets P[j] + (MF[i][j] * 0.7)$
			\ElsIf{$i = 2$} $P[j] \gets P[j] + (MF[i][j] * 0.2)$
			\Else{} $P[j] + (MF[i][j] * 0.1)$
			\EndIf
		\EndFor
	\EndFor
	\State $Q \gets \{P[j] \vert j \in 1 \dots n\}$ \Comment{distinct priorities}
	\State $G \gets array[1 \dots Card(Q)]$ \Comment{initially empty sets}
	\ForAll{$j \in 1 \dots n$}
		\State $p \gets P[j]$
		\State $G[p] \gets G[p] \cup \{j\}$
	\EndFor
	\State Sort every group in $G$ based on ascending execution time in $E$.
	\State Sort $P$ according to which group it belongs and its position within that group.
\end{algorithmic}
\end{algorithm}
