% !TeX root = ../../thesis.tex

\subsection{HGS}\label{ssec:alg-hgs}
The second algorithm was created by Harrold, Gupta and Soffa \cite{hgs}. This algorithm constructs the minimal hitting set of the test suite in an iterative fashion. As opposed to the greedy algorithm (\autoref{ssec:alg-greedy}), the HGS algorithm considers the test groups $CT$ instead of the set $TLt$ to obtain a list of test cases that cover all source code lines. More specifically, this algorithm considers the distinct test groups, denoted as $CTD$. Two test groups are considered indistinct if they differ in at least one test case. The pseudocode for this algorithm is provided in Algorithm \autoref{alg:hgs}.\\

\noindent Similar to the previous algorithm, an empty representative set $RS$ is constructed in which the selected test cases will be stored. The algorithm begins by iterating over every source code line $l \in C$ and constructing the corresponding set of test groups $CT_l$. As mentioned before, for performance reasons this set is reduced to $CTD$, only retaining distinct test groups. Next, the algorithm selects every test group of which the cardinality is equal to 1 and adds these to $RS$. This corresponds to every test case that covers a line of code, which is exclusively covered by that single test case. Subsequently, the lines that are covered by any of the selected test cases are removed from $C$. This process is repeated for an incremented cardinality, until every line in $C$ is covered. Since the remaining test groups will now contain more than one test case, the algorithm needs to make a choice on which test case to select. The authors have chosen that the test case that occurs in the most test groups is preferred. In the event of a tie, this choice is deferred until the next iteration.\\

\noindent The authors have provided an accompanying calculation of the computational time complexity of this algorithm \cite{hgs}. With respect to the naming convention introduced in \autoref{def:alg-naming}, additionally let $n$ denote the number of distinct test groups $CTD$, $nt$ the number of test cases $t \in TS$ and $MAX\_CARD$ the cardinality of the largest test group. The HGS algorithm consists of two steps which are performed repeatedly. The first step involves computing the number of occurrences of every test case $t$ in each test group. Given that there are $n$ distinct test groups and, in the worst case scenario, each test group can contain $MAX\_CARD$ test cases which all need to be examined once, the computational cost of this step is equal to $O(n * MAX\_CARD)$. In order to determine which test case should be included in the representative set $RS$, the algorithm needs to find all test cases for which the number of occurrences in all test groups is maximal, which requires at most $O(nt * MAX\_CARD)$. Since every repetition of these two steps adds a test case that belongs to at least one out of $n$ test groups to the representative set, the overall runtime of the algorithm is $O(n * (n + nt) * MAX\_CARD)$.
\begin{algorithm}[h!]
\caption{HGS algorithm (\cite{hgs})}
\label{alg:hgs}
\begin{algorithmic}[1]
	\State {\bfseries Input:} Distinct test groups $T_1, \dots T_n \in CDT$, containing test cases from $TS$.
	\State {\bfseries Output:} Subset $RS \subseteq TS$ of tests to execute.
	\State $marked \gets array[1 \dots n]$ \Comment{initially $false$}
	\State $MAX\_CARD \gets max \{Card(T_i) \vert T_i \in CDT\}$
	\State $RS \gets \bigcup \{ T_i \vert Card(T_i) = 1 \}$
	\ForAll{$T_i \in CDT$}
		\If{$T_i \cap RS \neq \emptyset$} $marked[i] \gets true$ \EndIf
	\EndFor
	\State $current \gets 1$
	\While{$current < MAX\_CARD$}
		\State $current \gets current + 1$
		\While{$\exists T_i : Card(T_i) = current, marked[i] = false$}
			\State $list \gets \{t \vert t \in T_i : Card(T_i) = current, marked[i] = false\}$
			\State $next \gets SelectTest(current, list)$
			\State $reduce \gets false$
			\ForAll{$T_i \in CDT$}
				\If{$next \in T_i$}
					\State $marked[i] = true$
					\If{$Card(T_1) = MAX\_CARD$} $reduce \gets true$ \EndIf
				\EndIf
			\EndFor
			\If{$reduce$}
				\State $MAX\_CARD \gets max \{Card(T_i) \vert marked[i] = false\}$
			\EndIf
			\State $RS \gets RS \cup \{next\}$
		\EndWhile
	\EndWhile
	
	\Function{SelectTest}{$size$, $list$}
		\State $count\gets array[1 \dots nt]$
		
		\ForAll{$t \in list$}
			\State $count[t] \gets |\{T_j \vert t \in T_j, marked[T_j] = false, Card(T_j) = size\}|$
		\EndFor
		
		\State $tests \gets \{t \vert t \in list, count[t] = max(count) \}$
		
		\If{$|tests| = 1$} \Return $tests[0]$
		\ElsIf{$|tests| = MAX\_CARD$} \Return $tests[0]$
		\Else{} \Return $SelectTest(size+1, tests)$
		\EndIf
	\EndFunction
\end{algorithmic}
\end{algorithm}