% !TeX root = ../../thesis.tex

\subsection{Dodona data}
As mentioned before, Dodona utilises the MiniTest testing framework in conjunction with SimpleCov to calculate the coverage. MiniTest will by default only emit the name of every failed test case, without any further information. Furthermore, SimpleCov can only calculate the coverage for the entire test suite and does not allow us to retrieve the coverage on a per-test basis. To answer the fourth research question and apply the \velocity{} predictors to Dodona, a Python script has been created to reconstruct the conditions of every failed test run. The script first queries the API of \githubactions{} to find which test runs have failed. This thesis will consider $\SI{120}{}$ failed runs. For every failed commit, the script retrieves the parent commit and calculates the coverage on a per-test basis. This thesis will assume that the coverage of the parent commit resembles the coverage of the failed commit. The coverage is calculated by applying the following two transformations to the parent commits and subsequently rescheduling these in \githubactions{}:

\begin{itemize}
	\item \textbf{Cobertura formatter:} The current SimpleCov reports can only be generated as HTML reports, preventing convenient analysis. We can resolve this problem by using the Cobertura formatter instead, which generates XML reports. The controller already supports the structure of these reports, as this formatter is commonly used by Java testing frameworks as well.
	
	\item \textbf{Parallel execution:} The Dodona test suite currently executes the test cases by four processes concurrently, to reduce the execution time. Every process individually records the code coverage, and at the end of the test suite, SimpleCov merges these separate reports into one. However, this process is not entirely thread-safe since the test suite requires shared resources. We do not require thread-safety to calculate the total coverage, but we do require this to generate the coverage on a per-test basis. As a result, parallel execution has been disabled.
	
\end{itemize}