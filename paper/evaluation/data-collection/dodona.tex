% !TeX root = ../../thesis.tex

\subsection{Dodona data}
As mentioned before, Dodona utilises the MiniTest testing framework in conjunction with SimpleCov to calculate the coverage. MiniTest will by default only emit the name of every failed test case, without any further information. Furthermore, SimpleCov is only capable of calculating the coverage for the entire test suite and does not allow to retrieve the coverage on a per-test basis. To answer the fourth research question and apply the \velocity{} predictors to Dodona, a Python script has been created to reconstruct every failed test run. This script first queries the API of \githubactions{} to find which test runs have failed. In this thesis, $\SI{62}{}$ failed runs have been used. For every failed commit, the script retrieves the parent commit and calculates the coverage on a per-test basis. This thesis will assume that the coverage of the parent commit resembles the coverage of the failed commit. The coverage is calculated by applying the following two transformations to the parent commits and subsequently rescheduling these in \githubactions{}:

\begin{itemize}
	\item \textbf{Cobertura formatter:} The current SimpleCov reports can only be generated as HTML reports, preventing convenient analysis. This problem is resolved by using the Cobertura formatter instead, which generates XML reports. The structure of these reports is already supported by the Controller.
	
	\item \textbf{Parallel execution:} To reduce the execution time, the test cases are concurrently executed by four processes. The code coverage is recorded per process and afterwards merged. However, SimpleCov is not entirely thread-safe. This is not a problem if the total coverage is required, but it does prevent the accurate generation of per-test coverage. As a result, parallel execution has been disabled.
	
\end{itemize}