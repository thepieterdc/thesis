% !TeX root = ../../thesis.tex

\subsection{Dodona build data}
As mentioned before, Dodona makes use of the MiniTest testing framework and the SimpleCov coverage tracker. By default, MiniTest only emits which test cases have failed without any further information. Furthermore, SimpleCov is only capable of calculating the coverage for the entire test suite and does not allow to retrieve the coverage on a per-test basis. In order to answer the fourth research question which analyses the performance of applying the \velocity{} predictors to Dodona, a Python script has been created to repeat every failed test run in a modified way to allow timing the execution and tracking the individual test case coverage. Essentially, this script queries the API of \githubactions{} to find failed test runs, \SI{62}{} failed runs were used in this thesis. For every failed commit, the script queries the API again to find the first successfully tested parent commit. These parent commits are used to obtain the coverage per test case on a codebase which resembles the failed commits as close as possible. After the appropriate parent commits have been identified, they are modified by applying the following two transformations and subsequently rescheduled in \githubactions{}:
\begin{itemize}
	\item \textbf{Cobertura formatter:} The currently used coverage formatter is only capable of generating a HTML report of the coverage, which is not desirable for analysis. The Cobertura formatter on the other hand is able to generate XML reports instead, these are already supported by the Controller and Java agent.
	
	\item \textbf{Disable parallel execution:} By default, the test cases are executed concurrently and divided over four processes. However, since SimpleCov is not thread-safe, this would render the tracked coverage invalid.
\end{itemize}