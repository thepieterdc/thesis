% !TeX root = thesis.tex

\chapter{Introduction}
\label{ch:introduction}
Given the complexity and rapid pace at which software is being built today, it is inevitable that at some point bugs will emerge. These bugs can either be introduced by a malfunctioning new feature, or by breaking existing functionality (\emph{a regression}). In order to detect bugs in an application before its customers do, an adequate \emph{testing infrastructure} is required.\\

\noindent This testing infrastructure consists of multiple \emph{test cases}, collectively referred to as the \emph{test suite} of the application. The quality of a test suite can be assessed in multiple ways. The first and most used option is to measure which fraction of the source code is tested by at least one test case, a ratio which is expressed as the \emph{coverage} of the application. Another possibility is to apply transformations to the source code and validate whether or not this results in a failed test case, a process indicated as \emph{mutation testing}.\\

\noindent Ideally, this testing process should be automated and performed after every change to the source code. This is generally a very time-consuming occupation, and as such has led to the creation of various automation frameworks and tools, collected under the name of \acrfull{ci}. Common examples of \acrshort{ci} practices are automatically running the test suite and estimating the code coverage after every pushed change to the \acrfull{vcs}.\\

\noindent However, applying these practices and maintaining a qualitative test comes at a cost. After every addition or modification to the source code, at least one new test case must be introduced to validate its correctness. As a result of the speed at which the source code tends to grow, the test suite suffers from severe scalability issues. While it is desired and ideally required to execute every single test case in the test suite, there are examples known to literature where this is not possible since this incurs an increasing delay in the development process, which in turn results in economic loss.\\

\noindent Three approaches can be taken towards resolving this issue by reducing the time occupied by waiting for the test results: \acrfull{tsm}, \acrfull{tcs} and \acrfull{tcp}. The main subject of this thesis will be to implement a framework for \acrshort{tcp}. To accomplish this, the next chapter will introduce important concepts which are used in modern software engineering. \autoref{chap:related-work} will elaborate on the aforementioned approaches and present accompanying algorithms. The implementation details of the new framework will be discussed in \autoref{chap:velocity}. Afterwards, \autoref{chap:evaluation} will evaluate the performance of this framework and provide insights into the characteristics of a typical test suite. More specifically, this chapter will research the probability of (repeated) test failure and the average duration of a test run. Finally, \autoref{chap:conclusion} will present additional ideas and improvements to the framework.