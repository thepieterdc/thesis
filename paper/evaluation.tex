\chapter{Results and evaluation}
(TODO)

1904.09416.pdf heeft een hele benchmark van 35.000.000 runs

- Experiment setup

  -  Data verzameling

  - Bespreek de geselecteerde projecten
  
- Resultaat van toepassing van alle algoritmes op alle projecten, met wat grafieken