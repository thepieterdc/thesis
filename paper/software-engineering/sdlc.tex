% !TeX root = ../thesis.tex

\section{Software Development Life Cycle}\label{sec:se-sdlc}
An implementation of the SDLC consists of two parts. The first part is a list of phases, and the second part is a function that describes the transitions between these phases. Depending on the nature of the software, existing phases can either be omitted or additional phases can be added. The five phases below were compiled from multiple sources \cite{2010govardhan, 7106435} and describe a generic approach to which most software projects adhere.
\begin{enumerate}
	\bolditem{Requirements phase} In the first phase of the development process, the developers acquaint themselves with the project and compile a list of the desired functionalities \cite{7106435}. Subsequently, the developers can decide on the financial details, the required hardware specifications as well as which external software libraries will need to be acquired.
	
	\bolditem{Design phase} After the developer has gained sufficient knowledge about the project requirements, they can use this information to construct an architectural design of the application. This design consists of multiple documents, such as user stories and UML-diagrams. A user story describes which actions can be performed by which users, whereas a UML-diagram specifies the technical interaction between the individual components.
	
	\bolditem{Implementation phase} In the third phase, the developers will write code according to the specifications defined in the architectural designs.
	
	\bolditem{Testing phase} The fourth phase is the most critical. This phase will require the developers and quality assurance managers to test the implementation of the application thoroughly. The goal of this phase is to identify potential bugs before the application is made available to other users.
	
	\bolditem{Operational phase} The final phase marks the completion of the project, after which the developers can integrate it into the existing business environment of their customer.
\end{enumerate}

\noindent After we have identified the phases, we must define the transition from one phase into another phase using a model. Multiple models exist in the literature \cite{2010govardhan}, with each model having its advantages and disadvantages. This thesis will consider the traditional model, which is still widely used as of today. The base of this model is the Waterfall model by Benington  \cite{united1956symposium}. Similar to a real waterfall, this model executes every phase in cascading order. However, this imposes several restrictions. The most prevalent issue is the inability to revise a design decision when performing the actual implementation. To mitigate this problem, Royce has proposed an improved version of the Waterfall model \cite{Royce:1987:MDL:41765.41801}, which does allow a phase to transition back to any preceding phase. \Cref{fig:waterfall-royce} illustrates this updated model.

\begin{figure}[htbp!]
	\includegraphics[width=\textwidth]{assets/tikz/waterfall-model.tikz}
	\caption{Improved Waterfall model by Royce}
	\label{fig:waterfall-royce}
\end{figure}

\noindent The focus of this thesis will be on the implementation and testing phases, as these are the most time-consuming phases of the entire process. The modification that Royce has applied to the Waterfall model is particularly useful when applied to these two phases in the context of \emph{software regressions} \cite{10.1007/978-3-540-77966-7_18}. We employ the term ``regression'' when a feature that was once working as intended is suddenly malfunctioning. The culprit of this problem can be a change in the code, but this behaviour can also have an external cause, such as a change in the system clock due to daylight saving time. Sometimes, a regression can even be the result of a change to another, seemingly unrelated part of the application code \cite{6588537}.

\subsection{Taxonomy of test cases}

Software regressions and other functional bugs can ultimately incur disastrous effects, such as severe financial loss or permanent damage to the reputation of the software company. The most famous example in history is without any doubt the explosion of the Ariane 5-rocket, which was caused by an integer overflow \cite{581900}. In order to reduce the risk of bugs, we should be able to detect malfunctioning components as soon as possible to warden the application against potential failures before they occur. Consequently, we must consider the testing phase as the most critical phase of the entire development process and therefore include sufficient test cases in the application. The collection of all the test cases in an application is referred to as the \emph{\gls{testsuite}}. We can distinguish many different types of test cases. This thesis will consider three categories in particular.

\subsubsection{Unit tests}
This is the most basic test type. The purpose of a unit test is to verify the behaviour of an individual component \cite{whittaker2000}. As a result, the scope of a unit test is limited to a small and isolated piece of code, e.g. one function. Implementation-wise, a unit test is typically an example of a \gls{whiteboxtest} \cite[p.~12]{6588537}. The term white-box indicates that the creator of the test case can manually inspect the code before constructing the test. As such, they can identify necessary edge values or corner cases. Common examples of these edge values include zero, negative numbers, empty arrays or array boundaries that might cause an overflow. Once the developer has identified the edge cases, they can construct the unit test by repeatedly calling the function under test, each time with a different (edge) argument value, and afterwards verifying its behaviour and result. These verifications are referred to as \emph{assertions}. \Cref{lst:se-unit-test} contains a unit test written in Java using the popular JUnit test framework.\\
	
\lstinputlisting[caption=Java unit test in JUnit., label=lst:se-unit-test, language=Java]{assets/listings/example-unit-test.java}

\clearpage

\subsubsection{Integration tests}
The second category involves a more advanced type of tests. An integration test validates the interaction between two or more individual components \cite{whittaker2000}. Ideally, accompanying unit tests should exist that test these components as well. As opposed to the previous unit tests, a developer will usually implement an integration test as a \gls{blackboxtest} \cite[p.~6]{6588537}. The difference between a black-box and a white-box test is that a black-box test assumes that the implementation details of the code under test are unknown during the construction of the test. Since a black-box test does not require any details about the code, we can, in fact, construct the integration tests before we implement the actual feature itself. A typical example of an integration test is the communication between the front-end and the back-end side of a web application. Another example is illustrated in \Cref{lst:se-integration-test}.\\
	
\lstinputlisting[caption=Java integration test in JUnit., label=lst:se-integration-test, language=Java]{assets/listings/example-integration-test.java}

\clearpage

\subsubsection{Regression tests}
After a developer has detected a regression in the application, they will add a regression test \cite[p.~372]{8016712} to the test suite. This regression test must replicate the exact conditions and sequence of actions that have triggered the failure. The goal of this test is to prevent similar failures to occur in the future if the same conditions would reapply. An example is provided in \Cref{lst:se-regression-test}.\\

\lstinputlisting[caption=Java regression test in JUnit., label=lst:se-regression-test, language=Java]{assets/listings/example-regression-test.java}