% !TeX root = ../../../thesis.tex

\subsubsection{Jenkins}
Jenkins CI\footnote{\url{https://jenkins.io/}} was started as a hobby project in 2004 by Kohsuke Kawaguchi, a former employee of Sun Microsystems. Jenkins is programmed in Java and profiles itself as ``The leading open source automation server''. It was initially named Hudson, but after Sun was acquired by Oracle, issues related to the trademark Hudson arose. In response, the developer community decided to migrate the Hudson code to a new repository and rename the project to Jenkins \cite{SmartJenkinsDefinitive}. As of today, Jenkins is still widely used for many reasons. Since it is open source and its source code is located on GitHub, it is free to use and can be self-hosted by the developers in a private environment. Furthermore, Jenkins provides an open ecosystem to support developers into writing new plugins and extending its functionality. A market research conducted by ZeroTurnaround in 2016 revealed that Jenkins is the preferred \CI{} tool by 60\% of the developers \cite{maple_2016}.

\begin{figure}[htbp!]
	\centering
	\includegraphics[width=0.45\textwidth]{assets/jenkins.pdf}
	\caption{Logo of Jenkins CI (\url{https://jenkins.io/})}
	\label{fig:jenkins}
\end{figure}