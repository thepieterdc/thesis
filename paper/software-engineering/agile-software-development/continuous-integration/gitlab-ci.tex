% !TeX root = ../../../thesis.tex

\subsubsection{GitLab CI}
GitLab, the main competitor of GitHub, announced its own \CI{} service in late 2012 named GitLab CI\footnote{\url{https://about.gitlab.com/blog/2012/11/13/continuous-integration-server-from-gitlab/}}. GitLab CI builds are configured using \emph{pipelines} and are executed by \emph{GitLab Runners}. These runners are operated by developers on their own infrastructure. Additionally, GitLab also offers the possibility to use \emph{shared runners}, which are hosted by themselves \cite{ciusinggitlab}. Equivalent to the aforementioned GitHub Actions, shared runners can be used for free by public repositories and are bounded by quota for private repositories \cite{gitlabdocs}. A downside of using GitLab CI is the lack of a community-driven plugin system, however this is a planned feature \footnote{\url{https://gitlab.com/gitlab-org/gitlab/issues/15067}}.

\begin{figure}[htbp!]
	\centering
	\includegraphics[width=0.40\textwidth]{assets/gitlab.pdf}
	\caption{Logo of GitLab CI (\url{https://gitlab.com/})}
	\label{fig:gitlab-ci}
\end{figure}