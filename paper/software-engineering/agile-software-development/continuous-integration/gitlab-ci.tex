% !TeX root = ../../../thesis.tex

\subsubsection{GitLab CI}
\gitlab{}, the main competitor to \github{}, announced its own \CI{} service in late 2012 named \gitlabci{}\footnote{\url{https://about.gitlab.com/blog/2012/11/13/continuous-integration-server-from-gitlab/}}. The build configuration is specified in a \emph{pipeline} and is executed by \emph{\gitlab{} Runners}. Developers may host these runners by themselves, or use shared runners hosted by \gitlab{}  \cite{ciusinggitlab}. Equivalent to the previously mentioned \githubactions{}, shared runners can be used for free by public repositories and are bound by quota for private repositories \cite{gitlabdocs}. A downside of using \gitlabci{} is the absence of a community-driven plugin system, but support for plugins is planned\footnote{\url{https://gitlab.com/gitlab-org/gitlab/issues/15067}}.

\begin{figure}[htbp!]
	\centering
	\includegraphics[width=0.40\textwidth]{assets/images/gitlab.pdf}
	\caption{Logo of GitLab CI (\url{https://gitlab.com/}).}
	\label{fig:gitlab-ci}
\end{figure}