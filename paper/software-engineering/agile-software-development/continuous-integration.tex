% !TeX root = ../../thesis.tex

\subsection{Continuous Integration}
In traditional software development, the design phase typically leads to a representation of the required functionality in multiple stand-alone modules. Subsequently, every module is implemented separately by individual developers. Afterwards, an attempt is made to integrate all the modules into the final application, an event to which Meyer refers to as the \emph{Big Bang} \cite[~p.103]{Meyer2014}. However, performing this operation has proven to be difficult, since every developer can take unexpected assumptions, which results in mutually incompatible components.\\

\noindent Contrarily, agile development methodologies advocate the idea of small, but frequent deliveries (\autoref{sssec:agilevalue-workingsoftware}). This in turn imposes that the integration of the modules is performed multiple times during the implementation phase on a \emph{continuous} basis, rather than just once at the end. 

% Microsoft ^
% https://www.researchgate.net/publication/5176165_Beyond_the_waterfall_software_development_at_Microsoft


%hier in hoofdstuk 7 staat iets over continuous integration
%https://link.springer.com/chapter/10.1007/978-3-319-05155-0_7

- changes moeten zeer regelmatig worden geintegreerd met elkaar -> feedback loop tussen implement -> integrate -> test -> repeat
- Continuous integration: wat?
- Bestaan aantal bestaande frameworks voor
- Maar; dat testen kan heel lang duren (zoek een bron waarin lange tests besproken worden)
- Bestaan aantal oplossingen voor -> zie volgende hoofdstuk

- feedback loop

- buildup naar waarom tooling nodig is

- waarom

- wat

- voorbeelden: Jenkins, CircleCI, Travis-CI, recent GitHub Actions + screenshots

- Probleem en oplossingen met regression tests

