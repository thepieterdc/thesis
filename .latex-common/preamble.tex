\documentclass[12pt,a4paper]{report}
\usepackage[utf8]{inputenc}
\usepackage[toc,page]{appendix}
\usepackage[english]{babel}
\usepackage[margin=1in]{geometry}
\usepackage[pdftex]{graphicx}
\usepackage{amsmath}
\usepackage{amssymb}
\usepackage{amsthm}
\usepackage{algorithm}
\usepackage{algpseudocode}
\usepackage{biblatex}
\usepackage{caption}
\usepackage{changepage}
\usepackage{enumitem}
\usepackage{hyperref}
\usepackage{listing}
\usepackage{listings,lstautogobble}
\usepackage[default,scale=0.95]{opensans}
\usepackage{scrlayer}
\usepackage{setspace}
\usepackage[caption=false]{subfig}
\usepackage{titlesec}
\usepackage{xcolor}

% Mathematical definitions.
\theoremstyle{break}
\newtheorem{definition}{Definition}
\newcommand{\definitionautorefname}{definition}

% Custom colors
\definecolor{code-background}{HTML}{EEEEEE}
\definecolor{code-delim}{RGB}{20,105,176}
\colorlet{code-punct}{red!60!black}

% Custom English translations.
\addto\extrasenglish{%
	\def\subsubsectionautorefname{section}%
}

% Configuration for code listings.
\lstdefinelanguage{json}{
	basicstyle=\normalfont\ttfamily,
	numbers=left,
	numberstyle=\scriptsize,
	stepnumber=1,
	numbersep=8pt,
	showstringspaces=false,
	breaklines=true,
	frame=lines,
	backgroundcolor=\color{code-background},
	literate=
	*{:}{{{\color{code-punct}{:}}}}{1}
	{,}{{{\color{code-punct}{,}}}}{1}
	{\{}{{{\color{code-delim}{\{}}}}{1}
	{\}}{{{\color{code-delim}{\}}}}}{1}
	{[}{{{\color{code-delim}{[}}}}{1}
	{]}{{{\color{code-delim}{]}}}}{1},
}


\newcommand\YAMLcolonstyle{\color{red}\mdseries}
\newcommand\YAMLkeystyle{\color{black}\bfseries}
\newcommand\YAMLvaluestyle{\color{blue}\mdseries}

\makeatletter

% here is a macro expanding to the name of the language
% (handy if you decide to change it further down the road)
\newcommand\language@yaml{yaml}

\expandafter\expandafter\expandafter\lstdefinelanguage
\expandafter{\language@yaml}
{
	keywords={true,false,null,y,n},
	keywordstyle=\color{darkgray}\bfseries,
	basicstyle=\YAMLkeystyle,                                 % assuming a key comes first
	sensitive=false,
	comment=[l]{\#},
	morecomment=[s]{/*}{*/},
	commentstyle=\color{purple}\ttfamily,
	stringstyle=\YAMLvaluestyle\ttfamily,
	moredelim=[l][\color{orange}]{\&},
	moredelim=[l][\color{magenta}]{*},
	moredelim=**[il][\YAMLcolonstyle{:}\YAMLvaluestyle]{:},   % switch to value style at :
	morestring=[b]',
	morestring=[b]",
	literate =    {---}{{\ProcessThreeDashes}}3
	{>}{{\textcolor{red}\textgreater}}1
	{|}{{\textcolor{red}\textbar}}1
	{\ -\ }{{\mdseries\ -\ }}3,
}

\lstset{
	numbers=left,
	numberstyle=\scriptsize,
	stepnumber=1,
	numbersep=8pt,
	showstringspaces=false,
	breaklines=true,
	frame=lines,
	backgroundcolor=\color{code-background},
	autogobble=true
}

\lstset{
	numbers=left,
	numberstyle=\scriptsize,
	stepnumber=1,
	numbersep=8pt,
	showstringspaces=false,
	breaklines=true,
	frame=lines,
	backgroundcolor=\color{code-background},
	autogobble=true
}

\titleclass{\chapter}{straight}

\titleformat{\chapter}[display]{\normalfont\huge\bfseries}{\chaptertitlename\ \thechapter}{18pt}{\huge}
\titlespacing*{\chapter}{0pt}{20pt}{20pt}

\setcounter{secnumdepth}{5}

% Default values for variables.
\newcommand{\documentdate}{\today}
\newcommand{\documenttitle}{Undefined "documenttitle"}

% Custom macros.
\newcommand{\bolditem}[1]{\item \textbf{#1:}}